\section{HYPOTHESIS AND EVALUATION}\label{Sec:concl}
As mentioned earlier in the paper, during evacuation of found humans in a fire outbreak, there poses a risk of worsening the case of the remaining humans who are stuck in the fire. Incorporating our robot in this scenario would identify humans left in the fire, using the LDR, while reducing the amount of fire still present and members of the rescue team take out found humans. The incoporation of this robot should improve overall rescue time as the fire fighter no longer need to restart search procedures after each evacuation as the robot constantly alerts them of the locations of humans, through the LDR, on the map. Furthermore, more lives would potentially be saved as reducing the time would reduce the effect of lack of oxygen due to the fire. \\

During fire fighting training, fire fighters often train in a controlled environment which is
set on fire in order to simulate actual scenarios. In this process, the fire fighters are expected to search for and evacuate all "dummy" humans captured in the burning area as well as put out the fire. To assess the impact the robots have on the fire fighting process, an experiment can be designed such that the fire fighting team would complete the controlled live-fire training scenario with and without the aid of the robot.To make the dummies identifiable by the LDR, heartbeats and respiratory signals of test humans would be prerecorded and put in a "transmitter", inserted into the body of the dummies. This would mimic the presence of humans to the LDR on the robot as it would detect those signals. Furthermore, speakers, which would expel human-like noises, would also be inserted into few of the dummies.\\

The robot is going to be evaluated based on a number of key performance indicators. This indicators essentially highlight the extent of aid the robot provides to the fire fighting team. These performance indicators are:
\begin{enumerate}
    \item Ability to detect survivors/victims: The robot should be able to identify all humans in the burning area
    \item Accuracy of the map of the area: The map has to be accurate enough as the locations of survivors would be placed on the map created by the robot
    \item Time taken to complete process with and without the aid of the robot: The time taken to complete the process with the robot should be considerably lower than without  using the robot.
    \item How effective is the robot in identifying and putting out fires
\end{enumerate}


% I will keep this document updated when new techniques are needed.  Google is always there for you to find the right way to write in $\LaTeX$, although you can also read any $\LaTeX$ Cheat Sheet, e.g.\\ \url{https://www.nyu.edu/projects/beber/files/Chang_LaTeX_sheet.pdf}
